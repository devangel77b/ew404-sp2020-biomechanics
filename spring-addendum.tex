\section*{Spring 2020 progress during the global COVID-19 pandemic}
In the Spring of 2020, as the global COVID-19 pandemic emerged, it became clear that the team's planned field mission to the California Channel Islands was not possible. Furthermore, with the Academy ordered to shift to remote instruction, there was no way to perform a local demonstration of the various stages of the original mission. As a result, the team decided to revisit several aspects from their concept and preliminary design reviews, and opted instead to take another trip around the design spiral, to improve the design further, for the benefit of future teams that attempt a field mission. Additional consideration was given to:
\begin{enumerate}
\item Further consideration of fixed wing and hybrid concepts to provide longer endurance during the survey phase of the mission (B.~Phan, \fref{sec:fixedwing})
\item Further consideration of machine vision aspects of the survey phase (J.~Kang, \fref{sec:vision})
\item Alternative quadrotor platforms to the DJI Matrice 100, due to DOD restrictions on DJI products (J.~Lim, \fref{sec:alternativequads})
\item Improvements to Matrice 100 hovering performance through additional sensors and control (J.~Lim, \fref{sec:hoveringcontrol})
\item Improved robot arm mechanical design (R.~Carroll and L.~Hofland, \fref{sec:robotarmmechanical})
\item Improved robot arm control design (J.~Kang, R.~Carroll and L.~Hofland, \fref{sec:robotarm})
\item Development of detailed operating and casualty procedures (J.~Kang, B.~Phan, and J.~Lim, appendices~\ref{app:OP} and \ref{app:CP})
\end{enumerate}

\subsection*{How STEVE could be used in response to COVID-19}
With the need for finding effective methods to mitigate COVID-19 pandemic spreading, there are several solutions that drones can offer during the crisis. Because STEVE is a platform that is aimed at carrying an extraction device capable of retrieving and securing plant samples from the origin and back, STEVE could serve public safety agencies in several aspects of the COVID-19 mitigation mission. By modifying the end effector and payload capacity, STEVE could retrieve and deliver medical supplies or testing kits to hard to reach places where hospitals or testing sites may not be readily accessible. STEVE could also return to the same destination to pick up the testing samples and return them to hospitals or testing sites for testing so that public safety employees don’t have to risk encountering those who may be affected by COVID-19. Furthermore, with computer vision, STEVE has the potential to monitor public spaces and identify large groups gathering and alert public safety agencies so that gatherings can be separated quickly. With agencies that already specialize in drone operations, STEVE could be quickly implemented for use without any large learning curve associated with it besides the end effector of the extraction device which could be modified for different missions in countering COVID-19.








